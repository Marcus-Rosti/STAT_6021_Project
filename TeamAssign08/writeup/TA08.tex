%%%%%%%%%%%%%%%%%%%%%%%%%%%%%%%%%%%%%%%%%
% University/School Laboratory Report
% LaTeX Template
% Version 3.1 (25/3/14)
%
% This template has been downloaded from:
% http://www.LaTeXTemplates.com
%
% Original author:
% Linux and Unix Users Group at Virginia Tech Wiki 
% (https://vtluug.org/wiki/Example_LaTeX_chem_lab_report)
%
% License:
% CC BY-NC-SA 3.0 (http://creativecommons.org/licenses/by-nc-sa/3.0/)
%
%%%%%%%%%%%%%%%%%%%%%%%%%%%%%%%%%%%%%%%%%

%----------------------------------------------------------------------------------------
%	PACKAGES AND DOCUMENT CONFIGURATIONS
%----------------------------------------------------------------------------------------

\documentclass{article}
\usepackage{fullpage,listings}
\usepackage[version=3]{mhchem} % Package for chemical equation typesetting
\usepackage{siunitx} % Provides the \SI{}{} and \si{} command for typesetting SI units
\usepackage{graphicx} % Required for the inclusion of images
\usepackage{natbib} % Required to change bibliography style to APA
\usepackage{amsmath} % Required for some math elements 

\setlength\parindent{0pt} % Removes all indentation from paragraphs

\renewcommand{\labelenumi}{\alph{enumi}.} % Make numbering in the enumerate environment by letter rather than number (e.g. section 6)

%\usepackage{times} % Uncomment to use the Times New Roman font

\usepackage{color}
 
\definecolor{codegreen}{rgb}{0,0.6,0}
\definecolor{codegray}{rgb}{0.5,0.5,0.5}
\definecolor{codepurple}{rgb}{0.58,0,0.82}
\definecolor{backcolour}{rgb}{0.95,0.95,0.92}
 
\lstdefinestyle{mystyle}{
    backgroundcolor=\color{backcolour},   
    commentstyle=\color{codegreen},
    keywordstyle=\color{magenta},
    numberstyle=\tiny\color{codegray},
    stringstyle=\color{codepurple},
    basicstyle=\footnotesize,
    breakatwhitespace=false,         
    breaklines=true,                 
    captionpos=b,                    
    keepspaces=true,                 
    numbers=left,                    
    numbersep=5pt,                  
    showspaces=false,                
    showstringspaces=false,
    showtabs=false,                  
    tabsize=2
}
 
\lstset{style=mystyle}
 

%----------------------------------------------------------------------------------------
%	DOCUMENT INFORMATION
%----------------------------------------------------------------------------------------

\title{Team Assignment 8 \\ Linear Regression \\ STAT 6021} % Title

\date{\today} % Date for the report

\begin{document}



\maketitle % Insert the title, author and date

\begin{center}
\begin{tabular}{l r}
Team members: & Don Chesworth \\ % Partner names
& Marcus Rosti \\
& Katherine Schinkel \\
& Mike Voltmer \\
Instructor: & Professor Holt % Instructor/supervisor
\end{tabular}
\end{center}

% If you wish to include an abstract, uncomment the lines below
% \begin{abstract}
% Abstract text
% \end{abstract}

%----------------------------------------------------------------------------------------
%	SECTION 1
%----------------------------------------------------------------------------------------

\section{Objective}
To analyze the effectiveness of merely subsetting parameters versus subsetting and transforming variables in a least squares regression setting.

% If you have more than one objective, uncomment the below:
%\begin{description}
%\item[First Objective] \hfill \\
%Objective 1 text
%\item[Second Objective] \hfill \\
%Objective 2 text
%\end{description}

\section{Question}
\subsection{Only Subsets}
To subset the data meaningfully, we used a mix of anova and the t test for significance.  We started by using regsubsets from the package leaps to identify the most important variables. This yielded age, PPE and DFA as the most important variables based on the exhaustive algorithm. Using those as a starting point, we next used anova to start removing variables. We dropped variables based on their significance and then verified there predictive power using 10 fold cross validation.  After getting to the point where we could no longer remove significant variables via ANOVA we were left with,
\begin{lstlisting}[language=R]
lm2 <- lm(motor_UPDRS ~ age + sex + test_time + Jitter.Abs. + NHR +
            Jitter.PPQ5 + Jitter.DDP + Shimmer.APQ3 + Shimmer.APQ5 
            + Shimmer.APQ11 + HNR + DFA + PPE, data=p)
\end{lstlisting}

\subsection{Subsets and transformations}
This was a more difficult task.  The variables had little meaningful transformations. We used a mix of visual inspection and cross validation to confirm the predictive and inferential power of the model.  Going off of only our subsets, we started adding transformations on the data and did cross validation to measure the fit.  This ultimately yielded an interesting model.  Based on cross validation, I raised age to the 13th and PPE to the 7th while removing NHR, PPE and sex. 
\begin{lstlisting}[language=R]
lm3 <- lm(motor_UPDRS ~ poly(age,13) + test_time + Jitter.Abs. +
            Jitter.PPQ5 + Jitter.DDP + Shimmer.APQ5 +
            Shimmer.APQ11 + HNR + DFA + poly(PPE,7), data=p)
\end{lstlisting}

\end{document}